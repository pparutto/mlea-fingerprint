La reconnaissance d'empreintes digitales consiste en une
reconnaissance d'un motif (empreinte sur le doigt) permettant
d'identifier le possesseur de cette empreinte. Il y a deux types
d'applications utilisant la reconnaissance d'empreintes :

\begin{description}
\item[Vérification] Une personne décline son identité et on vérifie
  que c'est vrai en comparant l'empreinte sur le doigt avec celle dans
  la base et on répond oui ou non.
\item[Identification] On n'a aucun indice à l'avance sur l'identité du
  possesseur de l'empreinte. Il faut donc comparer l'image à toutes
  celles de la base.
\end{description}

La phase d'acquisition, généralement commune aux deux types
d'applications, correspond à l'enregistrement de l'empreinte dans la
base de donnée. Elle est généralement suivie par une algorithme de
validation de la qualité de l'enregistrement comme le montre la Figure
\ref{fig:schema-bloc-acq}.

\begin{figure}[H]
\centering
\includegraphics[scale=0.8]{three_way.png}
\caption{Schéma bloc de l'acquisition, de la vérification et de
  l'identification.}
\label{fig:schema-bloc-acq}
\end{figure}

Comme on peut le voir sur la Figure, l'image est ensuite transformée
en un modèle (\emph{template} en anglais) grâce à un extracteur de
caractéristiques (\emph{feature extractor} en anglais).

Dans le cas de la vérification, l'utilisateur entre son identifiant et
donne son empreinte. Celle-ci est transformée en un modèle compact.
On extrait le modèle lui correspondant dans la base de donnée, et un
comparateur est chargé de prendre la décision ``oui / non''.

Dans le cas de l'identification, c'est différent car il n'y a aucun
identifiant qui est donné et ce n'est plus un simple écart entre deux
modèles à calculer, mais un écart avec toutes les données de la base.
Les résultats possibles sont l'identité du possesseur des empreintes,
ou ``utilisateur non identifié''.

Il y a différentes parties distinctes, et dans un premier temps, nous
allons traiter des problèmes liés à ce type de biométrie. Dans un
second temps, FIXME!!!


%%% Local Variables:
%%% mode: latex
%%% TeX-master: "../mlea"
%%% End:
