\section{Problèmes liés à la biométrie}

Il y a divers problèmes liés à cette biométrie, notamment lors de
l'acquisition. Il y a principalement deux moyens d'effectuer ce
processus \emph{off-line}, c'est à dire la méthode ancestrale avec de
l'encre sur le doigt et apposé sur du papier, et \emph{live} qui
correspond à utiliser un scanner numérique. Pour assurer une
compatibilité entre ses deux méthodes d'acquisition, le ``US Criminal
Justice Information Services'' a présenté un ensemble de
spécifications pour fixer la qualité et le format des images (voir
annexes F et G de \cite{nla.cat-vn4185009}).

Bien que la méthode à l'encre ait commencé à être utilisé il y a très
longtemps (plus d'une trentaine d'année) elle est toujours utilisée
dans des applications à caractères légales. Les bases de données des
agences américaines de sécurités contiennent des données ayant été
acquises par les deux méthodes, ce qui signifie que les algorithmes
travaillant sur ces bases doivent être capables de les traiter
indépendemment.

Un des avantages d'utiliser l'acquisition avec l'encre est que l'on
peut enregistrer tout le doigt en effectuant un mouvement rotatif en
partant avec son doigt perpendiculaire d'un côté pour aller à
perpendiculaire de l'autre, ce qui n'est pas possible avec un scanner
numérique.

En contrepartie, il est possible, en fonction de comment est placée
l'encre sur le doigt, de laisser des zones de vide ou alors trop
pleine.


Du côté des acquisitions numériques, la plupart des appareils peuvent
se classer dans trois catégories différentes :

\begin{description}
\item[Optiques] Le plus vieux et le plus utilisé des techniques
  d'acquisitions actuelles. Cette méthode fonctionne avec un prisme en
  verre. L'utilisateur doit mettre son doigt en haut de ce prisme est
  l'acquisition se fait ainsi.

  Ce type d'appareil a l'avantage de faire des images de bonnes
  qualités et d'avoir une grande surface du doigt enregistrée. Par
  contre, on ne peut pas miniaturiser cet appareil.
\end{description}

%%% Local Variables:
%%% mode: latex
%%% TeX-master: "../mlea"
%%% End:
