En conclusion, la reconnaissance d'empreinte digitales est une méthode
qui obtient de très bons résultats. Dans la partie
\ref{sec:performances}, on voit que l'on peut obtenir un taux de vrai
acceptation de 97.6\% avec la méthode des Minutia. D'autres méthodes
plus compliquées existent obtenant de meilleur performances mais sont
souvent plus coûteuses.\\

La méthode de calcul de similarité entre deux minutiae présenté repose
sur la distance d'édition entre deux chaînes de caractères qui est une
méthode classique et très étudié. Toute l'élégance de la méthode est
la transformation des informations de crêtes en chaînes de caractères
pouvant facilement êtres comparées. La comparaison sur une base de
données présenté dans la section~\ref{sec:performances} montre que
cette méthode donne de bons résultats.\\

Si nous avions à implémenter la méthode qui nous donnerait les
meilleures performances, nous choisirions celle des Minutia. Cependant
c'est la méthode la plus difficile à implémenter, car il faut détecter
les Minutia, les épurer, puis le travail de reconnaissance commence,
tandis que dans la méthode basée sur la corrélation FIXME.



%%% Local Variables:
%%% mode: latex
%%% TeX-master: "../mlea"
%%% End:
